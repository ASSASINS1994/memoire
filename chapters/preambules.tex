%!TEX root = ../main.tex

\begin{center}
    \normalsize{Ministère de l'Enseignement Supérieur et de la Recherche Scientifique}\\
    \normalsize{Université des Sciences et de la Technologie Houari Boumediene}\\
    \normalsize{Faculté d'électronique et d'informatique}\\
    \normalsize{Département d'informatique}\\
    \end{center}
    \begin{center}
    \includegraphics[width=4cm,height=4cm]{images/USTHB_Logo.png}
    \end{center}


    \begin{center}
    \Huge{\textbf{Mémoire de Licence}}\\
    \large{Domaine Informatique}\\
    \textbf{}\\
    \large{\textbf{Option: Informatique}}\\
    \textbf{}\\
    \bigskip
    \normalsize{\textbf{Thème}}
    \end{center}
    \shabox{
    \begin{minipage}{0.9\textwidth}
    \begin{center}
    \Large{Étude technique et développement de malware}
    \end{center}
\end{minipage}
}
\\
\\
\\
\begin{table}[h]
    \center
    \begin{tabular}{p{8cm}p{6.5cm}}
    \textbf{Présenté par :} & \textbf{Sujet proposé par :}\\
    - \textsc{Bitam} Salim  & - \\
    - \textsc{Malaoui} Sidahmed Bilal  & \\
    \end{tabular}
\end{table}
\\
\\

\begin{table}[h]
    \begin{tabular}{p{6.5cm}p{5cm}}
    \textbf{Devant le jury composé de:}&\\
    - \,\,\, Mr. ******* & Président \\
    - \,\,\, Mme. ****** & Membre \\
    \end{tabular}
\end{table}

\vspace{2cm}
\begin{center}
Projet N : ***/2016
\end{center}

\clearpage
% \begin{titlepage}
%     \begin{center}
%         \vspace*{1cm}
        
%         \Huge
%         \textbf{Étude technique et développement de malware}
        
%         \vspace{0.5cm}
%         \Large
        
%         \vspace{1.5cm}
        
%         \begin{tabular}{cc}
    
%             % La commande \hline affiche une ligne horizental (pour séparer les lignes). On utilise cette commande partout ou on souhaite obtenir une 
%             % ligne horizental dans le tableau.
%             % On sépare le contenu de chaque cellule par '&', et chaque ligne du tableau doit se terminer par '\\'.
%             \textbf{Malaoui Sidahmed Bilal} & \textbf{Bitam Salim} \\
%             \texttt{\normalsize{sidahmed.malaoui@gmail.com}} & \texttt{\normalsize{salim.bitam@outlook.com}} \\

%         \end{tabular}
%         % \textbf{Malaoui, Sidahmed Bilal  \hfill \vcenter Bitam, Salim\\
%                 % \large{\texttt{sidahmed.malaoui@gmail.com} \hfill \texttt{salim.bitam@outlook.fr}}
%         % }
%         \vfill
        
%         Une thère présentée pour le grade de\\
%         Licence en Informatique
%         % A thesis presented for the degree of\\
%         % Doctor of Philosophy
        
%         \vspace{0.8cm}
        
%         \includegraphics[width=0.4\textwidth]{images/USTHB_Logo.png}
        
%         \LARGE
%         Département Informatique\\
%         Université des Sciences et de la Technologie Houari Boumediene\\
%         Algérie\\
%         \today
        
%     \end{center}
% \end{titlepage}

% \maketitle

\pagenumbering{roman} %Utilisé la numérotation romaine des pages.
% Pour abstract
\thispagestyle{plain}
\begin{center}
    \Large
    \textbf{Étude technique et développement de malware}
    
    \vspace{0.4cm}
    \large
    % Thesis Subtitle
    
    \vspace{0.4cm}
    \textbf{Malaoui Sidahmed Bilal \hspace{3cm} Bitam Salim}
    
    \vspace{0.9cm}
    \textbf{Abstract}
\end{center}
% \chapter*{Abstract}
% Abstract goes here

\chapter*{Dédicaces}
Dédicaces ici.

\chapter*{Remerciements}
Les remerciements ici.