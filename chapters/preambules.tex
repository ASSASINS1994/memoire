%!TEX root = ../main.tex

\begin{center}
    \normalsize{Ministère de l'Enseignement Supérieur et de la Recherche Scientifique}\\
    \normalsize{Université des Sciences et de la Technologie Houari Boumediene}\\
    \normalsize{Faculté d'électronique et d'informatique}\\
    \normalsize{Département d'informatique}\\
    \end{center}
    \begin{center}
    \includegraphics[width=4cm,height=4cm]{images/USTHB_Logo.png}
    \end{center}


    \begin{center}
    \Huge{\textbf{Mémoire de Licence}}\\
    \large{Domaine Informatique}\\
    \textbf{}\\
    \large{\textbf{Option: Informatique}}\\
    \textbf{}\\
    \bigskip
    \normalsize{\textbf{Thème}}
    \end{center}
    \shabox{
    \begin{minipage}{0.9\textwidth}
    \begin{center}
    \Large{Étude et développement de malware}
    \end{center}
\end{minipage}
}
\\
\\
\\
\begin{table}[h]
    \center
    \begin{tabular}{p{8cm}p{6.5cm}}
    \textbf{Présenté par :} & \textbf{Sujet proposé par :}\\
    - \textsc{Bitam} Salim  & - \textsc{Zeraoulia} Khaled \\
    - \textsc{Malaoui} Sidahmed Bilal  & \\
    \end{tabular}
\end{table}
\\
\\

\begin{table}[h]
    \begin{tabular}{p{6.5cm}p{5cm}}
    \textbf{Devant le jury composé de:}&\\
    - \,\,\, Mr. \textsc{Belkhir} & Président \\
    - \,\,\, Mme. \textsc{Chenait} & Membre \\
    \end{tabular}
\end{table}

\vspace{2cm}
\begin{center}
Projet N : 217/2016
\end{center}

\clearpage

\pagenumbering{roman} 

\section*{Résumé} 
L’information est aujourd’hui devenue la plus grande richesse que toute personne ou entreprise peut posséder, et la sensibilité de ces mêmes informations a fait de la sécurité informatique une priorité majeure des individus et des professionnels.

Mais la naïveté de l’être humain l’a poussé à penser que sécuriser sa machine est amplement suffisant pour être à l’abri des attaques, il oublie cependant que sa machine est très souvent en interaction avec d’autres machines de sécurité moindre.

Dans cette étude, nous tenterons de mieux comprendre le fonctionnement des Virus, leurs méthodes d'infection et de propagation. Au vu de développer un Malware combinant un virus et une porte dérobée puis de tester sa propagation d'une machine de faible sécurité vers une machine sécurisée, et de tester son efficacité contre cette dernière. %vérifié

\section*{Abstract}
The information has now become the greatest asset that any person or company can own , and the sensitivity of this information has made computer security a major priority for individuals and professionals.

But the naivety of the human being led him to think that securing his own machine is  enough to be safe from attack , however, he forgets that his machine is very often in interaction with other lesser security machines.

In this study, we will try to better understand how the virus works , and their methods to infect and to propagate. In order to develop a Malware, combining a Virus and a Backdoor then test its spread from a low security machine to a secured machine and test its effectiveness against the latter. %vérifié

\paragraph{Mots clés :} Malware, Virus, Backdoor, Infection virale, Propagation.


\chapter*{Dédicaces}
Je dédie ce travail à mes parents qui m'ont toujours supporté et qui ont toujours sacrifié pour moi. Tout ce que
je pourrai faire pour eux n'atteindre même pas le quart de ce qu'ils m'ont donné et de ce qu'ils m'ont fait.

À ma chère famille, surtout mon petit frère et ma sœur.

À mes chers amis, Nassraddine, Lamine, Idris, Rabah, Habib, Walid, Hawas.

J'aimerai aussi et surtout dédié ce modeste travail à mes chers frères et sœurs Palestiniens qui sont opprimés par les
sionistes, les sionistes qui n'ont pas la moindre particule de noblesse ou de courage. Je dis au peuple Palestinien
que ça me fait mal au cœur de vous voir souffrir et de ne rien pouvoir faire, et ça me fait encore plus mal que 
les gouvernements qui prétendent être musulmane vous faite du mal et conspirent contre vous au-lieu de vous aider, tous 
ça pour gagner la bénédiction de leurs maîtres. 
Mais bi idni Allah, nous continuerons de nous développer pour pouvoir vous être utile dans un future proche.

Je dédie aussi ce travail aux peuples opprimés à travers le monde entier, peu importe leurs religions, peu importe
leurs origines. L'injustice reste de l'injustice, peu importe qui est l'oppresseur, peu importe qui est l'oppressé.

\vfill
Sidahmed
\vfill
\newpage
Je dédie ce travail à mes très cher parents qui m'ont toujours supporté dans mes moment difficile et qui se sont toujours sacrifié pour moi, à mon merveilleux père qui ma offert aide, patience et conseils, et ma merveilleuse mère dont les mots ne peuvent décrire, qui ma apporter un soutien morale et des conseils inestimable.

À ma chère famille, dont mes sœurs  adorées

À mes chers amis, Zaki, Ahmed, Sidahmed.

J'aimerai aussi et surtout dédié ce modeste travail à mes chers frères et sœurs Palestiniens qui sont opprimés par les
sionistes, les sionistes qui n'ont pas la moindre particule de noblesse ou de courage. Je dis au peuple Palestinien
que ça me fait mal au cœur de vous voir souffrir et de ne rien pouvoir faire, et ça me fait encore plus mal que 
les gouvernements qui prétendent être musulmane vous faite du mal et conspirent contre vous au-lieu de vous aider, tous 
ça pour gagner la bénédiction de leurs maîtres. 
Mais bi idni Allah, nous continuerons de nous développer pour pouvoir vous être utile dans un future proche.

Je dédie aussi ce travail aux peuples opprimés à travers le monde entier, peu importe leurs religions, peu importe
leurs origines. L'injustice reste de l'injustice, peu importe qui est l'oppresseur, peu importe qui est l'oppressé.
\vfill
Salim
\vfill

\chapter*{Remerciements}
Nous voudrons commencer par remercier \textsc{dieu}, le miséricordieux, qui nous a donnée les moyens et la force
d'aller jusqu'au bout de ce projet.

Nous sommes très reconnaissants à Monsieur \textsc{Belkhir} et Madame \textsc{Chenait} d'avoir accepté de juger notre travail.

Nous tenons à remercier notre promoteur Monsieur \textsc{Zeraoulia} Khaled pour son soutient.

Un très grand merci à Monsieur \textsc{Adda} Mohamed pour son soutient et son aide précieux.

Nous voudrons aussi remercier le département de l'informatique d'ELIT, de nous avoir accordé un moment précieux de
leurs temps inestimable.

Enfin, nous remercions tous ceux qui ont contribué de près ou de loin à la réalisation de ce projet.
