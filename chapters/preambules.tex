
\begin{titlepage}
    \begin{center}
        \vspace*{1cm}
        
        \Huge
        \textbf{Étude technique et développement de malware}
        
        \vspace{0.5cm}
        \Large
        
        \vspace{1.5cm}
        
        \begin{tabular}{cc}
    
            % La commande \hline affiche une ligne horizental (pour séparer les lignes). On utilise cette commande partout ou on souhaite obtenir une 
            % ligne horizental dans le tableau.
            % On sépare le contenu de chaque cellule par '&', et chaque ligne du tableau doit se terminer par '\\'.
            \textbf{Malaoui Sidahmed Bilal} & \textbf{Bitam Salim} \\
            \texttt{\normalsize{sidahmed.malaoui@gmail.com}} & \texttt{\normalsize{salim.bitam@outlook.com}} \\

        \end{tabular}
        % \textbf{Malaoui, Sidahmed Bilal  \hfill \vcenter Bitam, Salim\\
                % \large{\texttt{sidahmed.malaoui@gmail.com} \hfill \texttt{salim.bitam@outlook.fr}}
        % }
        \vfill
        
        Une thère présentée pour le grade de\\
        Licence en Informatique
        % A thesis presented for the degree of\\
        % Doctor of Philosophy
        
        \vspace{0.8cm}
        
        \includegraphics[width=0.4\textwidth]{images/USTHB_Logo.png}
        
        \LARGE
        Département Informatique\\
        Université des Sciences et de la Technologie Houari Boumediene\\
        Algérie\\
        \today
        
    \end{center}
\end{titlepage}

% \maketitle

\pagenumbering{roman} %Utilisé la numérotation romaine des pages.
% Pour abstract
\thispagestyle{plain}
\begin{center}
    \Large
    \textbf{Étude technique et développement de malware}
    
    \vspace{0.4cm}
    \large
    % Thesis Subtitle
    
    \vspace{0.4cm}
    \textbf{Malaoui Sidahmed Bilal \hspace{3cm} Bitam Salim}
    
    \vspace{0.9cm}
    \textbf{Abstract}
\end{center}
% \chapter*{Abstract}
% Abstract goes here

\chapter*{Dédicaces}
Dédicaces ici.

\chapter*{Remerciements}
Les remerciements ici.