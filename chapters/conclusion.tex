%!TEX root = ../main.tex

Ce projet de fin d'études a été riche en enseignements, qui nous a introduit dans le monde de la sécurité informatique et nous a permis en tant qu'étudiants d'apprendre les principes de base de la discipline tout cela en développant un Malware hybride.

Tout d'abord, nous avons introduit quelques notions de base sur la sécurité et les Malwares, puis nous nous sommes étalés sur les techniques d'infection virale.

En second lieu, notre travail a consisté à concevoir et à mettre en œuvre un Malware hybride, dont la tâche est d'infecter les exécutables contenus dans les périphériques de stockage externe d'une machine en s'injectant dedans et de fournir une porte dérobée pour un futur accès distant à la machine. Son développement a nécessité l'apprentissage des langages de programmation dont le C, le Python et bien sûr, l'assembleur, nous avons aussi appris a manipulé les logiciels de désassemblage tels que : IDA et Ollydbg.

Finalement, nous avons simulé un environnement d'une entreprise, où une machine qui présente une vulnérabilité a été exploité et infecter par notre Malware, donnât que cette machine communique en interne avec les maillons fort, cela a mené à la propagation de ce dernier, et l'infection des machines de l'entreprise.

De ce fait, nous avons démontré que la sécurité de chaque machine dépend de celle  avec qui elle communique.

Dans notre processus de travail, nous avons appris le travail de groupe et la répartition des tâches, aussi des connaissances de qualité qui étaient nouvelles pour nous ont été acquise. À partir de là, nous avons pu contempler le monde de la sécurité informatique, ce qui nous a poussé à vouloir poursuivre notre parcours universitaire et professionnel dans le domaine de la \emph{Sécurité des Systèmes d'Informations}.
