%!TEX root = ../main.tex

% \begin{itemize}
%     \item Parler sur l'informatique, son évolution, et son rôle dans les entreprise.
%     \item L'importance de l'information dans les entreprise.
%     \item La sécurité informatique
% \end{itemize}
% Le monde dans lequel nous vivons évolue constamment, apportant chaque année son lot de nouveautés, et ces nouveautés
% ont besoin de plus en plus de travail pour en tirer profit. C'est pour cette raison que l'informatique est né

% Il ne fait désormais plus aucun doute que l'informatique est devenu un domaine 
Il ne fait désormais plus aucun doute que les technologies de l'information et de la communication représentent 
la révolution la plus importante et la plus innovante qui a marqué la vie de l'humanité en ce siècle passé. 
En effet, elles viennent nous apporter de multiples conforts à notre mode de vie en révolutionnant le 
travail des individus par leur capacité de traitement d'information

De ce fait, l'informatique est devenue un outil incontournable de gestion, d'organisation, de production et de communication. Le réseau informatique des entreprises mettent en œuvre des données sensibles, les stocke, les partage en interne, les communique parfois à d'autres entreprises ou personnes ou les importe à partir d'autres sites. Cette ouverture vers l'extérieur conditionne des gains de productivité et de compétitivité.

Il est donc impossible de renoncer aux bénéfices de l'informatisation, d'isoler le réseau de l'extérieur, de retirer aux données leur caractère électronique et confidentiel. Les données sensibles du système d'information de l'entreprise sont donc exposées aux actes de malveillance dont la nature et la méthode ne cesse de changer. %Les prédateurs et voleurs s'attaquent aux ordinateurs surtout par le biais d'accès aux réseaux qui relient l'entreprise à l'extérieur.

% La sécurité du système d'information d'une entreprise est un requis important pour la poursuite de ses activités. Qu'il s'agisse de la dégradation de son image de marque, du vol de ses secrets de fabrication ou de la perte de ses données clients ; une catastrophe informatique a toujours des conséquences fâcheuses pouvant aller jusqu'au dépôt de bilan. %On doit réfléchir à la mise en place d'une politique de sécurité avant même la création du réseau. 

Cela a poussé les administrateurs des systèmes informatisés à dépenser des sommes colossales pour mettre à niveau 
des mesures de sécurités permettant d'arrêter les intrusions externes. Par contre, la majorité des ces administrateurs omis le fait qu'une attaque peut venir de l'intérieur.% Mais cela les a poussé aussi à ommettre 
% la possibilité qu'une attaque peut venir de l'intérieur.
% Le problème qui se pose est que les entreprise dépensent des sommes colossales pour se sécurisé de l'extérieur, mais 
% ils ommettent qu'une attaque peut toujours venir de l'intérieur. 
 À quoi bon avoir une porte 
blindé avec une verrouille électronique qui vérifient les empreintes digitales, s'il y a 
une simple porte en bois derrière qui accueille chaleureusement le cambrioleur. Si on généralise ce problème
sur la sécurité informatique, à quoi bon avoir un système de détection d'intrusion sophistiqué, et limités
les extension des 
fichiers qui puissent être téléchargé, et \ldots{} etc, si un personnel imprudent puisse utiliser dans sa machine de la 
société une clé USB qui coutoie quotidiennement son ordinateur de maison rassasié de virus. %peut ramener avec lui une clé 
% USB infecté et l'utiliser sur les machines de l'entreprise.


% Donc toute entreprise qui se respecte mettent des mesures de sécurités permettant de stopper les intrusions externes, 
% et dépensent des sommes colossale pour en arriver.
% Mais le fait qu'il faut à tout prix sécurisé les communications externent a poussé les entreprises et les administrateurs
% des systèmes informatisé à omettre le fait qu'une attaque peut venir de l'intérieur, et qu'une attaque venant de 
% l'intérieur peut provoquer des dégâts néfastes. Il sont 

% Mais la naïveté de l'être humain l'a poussé à penser que sécurisé
% Mais la naïveté des entreprises et des administrateurs des systèmes informatisés les a poussés à penser que sécuriser
% leurs machines des attaques externes est amplement suffisant pour sécuriser leurs systèmes. Alors que les études ont
% montrer que 93\% des problèmes accurant dans les entreprises viennent de l'intérieur, alors que seulement 7\%
% des attaques 


% L'objectif principal de cette étude 

% Le second objectif

Le premier chapitre de ce mémoire portera sur certaines notions et concepts de base de la sécurité informatique
, dont les \emph{programmes malveillants} et le \emph{hacking} font partie, dans le but de donner une terminologie et un vocabulaire spécifique à ce domaine d'activité.

Le second chapitre sera plus poussé et plus spécifique, et on y parlera des virus et plus précisément
des différentes techniques utilisées par ces derniers dans le processus d'infection, ainsi que les 
techniques qu'ils emploient pour dissimuler leurs présences. 
% Ces notions sont indispensable pour la
% réalisation de l'étude et l'obtention du résultat voulu.
%virifé

% Finalement, le dernier chapitre sera le fruit de toute les études et progrès faites durant le deux premiers chapitres.
% Ce chapitre consistera à attendre une machine \emph{A} qui est supposé être sécurisé, en exploitant et injectant le 
% logiciel malveillant dans une autre machine \emph{B} qui est en

% Finalement, le dernier chapitre de ce mémoire modélisera toute les compétences et les informations acquises dans les 
% chapitres précédents. La modélisation sera  
Ensuite, le dernier chapitre de ce mémoire présentera l'environnement de travail mis en place afin de faire 
propager le virus jusqu'à une machine sécurisé, et y obtenir un accès total sans déclencher d'alerte.

La succès de cette opération démontrera indéniablement que la sécurité d'une machine dépend toujours
de la sécurité des machines avec lesquelles elle interagit, et ainsi, il ne faut jamais sous-estimé les risques qui puisse être causé de l'intérieur.
% Nous allons commencer par envoyer le virus vers une machine par le biais d'une vulnérabilité, cette machine quotie 
% quotidiennement une autre machine sécurisé

% le programme malveillant. l'injecter par réseau dans une machine par le biais d'une vulnérabilité, et 

