%!TEX root = ../main.tex

L’informatique est devenue un outil incontournable dans les domaines d’activités scientifiques, techniques, industriels et autres, dont les champs d’application ne cessent de proliférer au sein de nos sociétés.

Pour les entreprises, la mise en place d’une bonne infrastructure informatique est un excellent moyen pour améliorer son organisation, son stockage de données, et même sa productivité. 

L’informatique permet  d’accroître l’efficacité opérationnelle d’une société, en permettant d’améliorer sa réactivité.

D’un autre côté, l’utilisation à outrance de cette technologie au service des domaines d’activités sus-cités a 
ouvert la porte à de nouvelles menaces dont la plus importante provient de l’internet sous formes de cyber 
attaques commises pour détruire, altérer, accéder à des données sensibles dans le but de les modifier ou de nuire au 
bon fonctionnement des réseaux : les motivations sont diverses et fonction de la nature des informations recherchées 
et de l’organisme visé \ldots

Cela a poussé les administrateurs des systèmes informatisés à dépenser des sommes colossales afin de protéger et mettre 
à niveau des mesures de sécurité permettant d’arrêter les intrusions externes.

Par contre, la majorité de ces administrateurs ont omis le fait qu’une attaque peut venir de l’intérieur. 
À quoi bon avoir une porte blindée avec un verrouillage électronique qui vérifie les empreintes digitales, 
s'il y a derrière une simple porte en bois qui accueille le cambrioleur. 

Si on généralise cet exemple sur la sécurité informatique, à quoi bon avoir un système de détection d’intrusion 
sophistiqué, si un personnel imprudent utilise dans les machines de l'entreprise une clé USB qui côtoie 
quotidiennement son ordinateur de maison lequel peut facilement être infecté.

Le principal objectif de la présente étude sera donc l'apprentissage des techniques d'infection et de propagation 
des virus. Puis de développer un Malware combinant un virus et une porte dérobée, en vue de son utilisation pour 
réaliser un test de propagation et d'obtention d'accès  à une machine sécurisée, par le biais d'un maillon 
faible de l'environnement  de travail mis en place.

Par conséquent, le premier chapitre de ce mémoire portera sur certaines notions et concepts de base de la 
sécurité informatique, dont les programmes malveillants et le hacking font partie, dans le but de donner une 
terminologie et un vocabulaire spécifique à ce domaine d’activité.

Le second chapitre sera plus poussé et plus spécifique, on y parlera des virus
et plus précisément des différentes techniques utilisées par ces derniers dans le processus d’infection, 
ainsi que les techniques qu’ils emploient pour dissimuler leurs présences.

Le dernier chapitre de ce mémoire présentera l'environnement de travail
mis en place  dans le cadre de ce projet, afin de propager le virus jusqu’à une machine sécurisé, et y obtenir un 
accès total sans déclencher d’alerte.

Le succès de cette opération démontrera indéniablement que la principale menace contrairement à ce que l’on pourrait 
penser, ne vient pas de l'extérieur, mais de l'intérieur ; que la sécurité d’une machine dépend  de la 
sécurité des machines avec laquelle elle interagit ; sans oublier que le maillon faible de la sécurité 
informatique est souvent le facteur humain, en d'autre terme, l'utilisateur.