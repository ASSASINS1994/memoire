%!TEX root = ../main.tex

Le premier chapitre de ce mémoire portera sur quelques notions et concepts de base de la 
sécurité informatique, à l’effet d’éclaircir et de définir une terminologie et un vocabulaire 
spécifique à ce domaine d’activité. %vérifié

\newpage

\section{Généralités sur la sécurité des systèmes informatiques}
    \subsection{Introduction}
    Dans les temps anciens, les besoins des gens étaient simples : nourriture, eau, abri, et la chance occasionnelle
    de propager l'espèce. Leurs besoins de base n'ont pas changé, mais les façons dont elles sont comblées ont changé.
    La nourriture est achetée dans des magasins qui sont alimentés par des chaînes d'approvisionnement avec des
    systèmes d'inventaire informatisés ; l'eau est distribuée à travers des systèmes contrôlés par des ordinateurs ; 
    les abris sont achetés et vendus par des agents immobiliers maniant des ordinateurs.
    La production et la transmission d'énergie pour faire fonctionner tous ces systèmes est contrôlé par des
    ordinateurs, et c'est les ordinateurs qui sont utilisés pour gérer les transactions financières pour le 
    payement de tout cela. \cite{virus} %vérifier

    Ce n'est pas un secret que l'infrastructure de la société repose sur des ordinateurs maintenant.
    Malheureusement, cela signifie qu'une menace envers les ordinateurs est une menace envers la société. 
    Mais quels sont les problèmes auxquels les infrastructures essentielles et critiques sont-elles confrontées.%vérifié

    \subsection{Types de menaces}
    Il y a quatre principales menaces à considérer. Ce sont les quatre cavaliers de l'apocalypse électronique :%vérifié
    \begin{description}
        \item[Spam :] Le terme couramment utilisé pour décrire les courriers électroniques non sollicités envoyés 
            en grand nombre à des boîtes aux lettres électroniques ou à des forums, dans un but publicitaire 
            ou commercial. Les statistiques varient au fil du temps, mais suggèrent que plus de 
            70\% \cite{spam} du trafic des courriers électroniques tombe dans cette catégorie.%vérifié
        \item[Bugs :] Ce sont des erreurs logicielles qui, quand elles surgissent, peuvent faire crasher votre 
            logiciel immédiatement, si vous êtes chanceux. Elles peuvent également entraîner une corruption de données,
            des failles de sécurité, et des problèmes très difficiles à trouver. \cite{virus} %vérifié
        \item[Dénis de service (DoS):] Ce type de menace affame l'utilisation légitime des ressources ou des 
            services. Par exemple, une attaque DoS peut utiliser tout l'espace disque disponible sur un 
            système, de sorte que les autres utilisateurs ne pourront pas faire usage de celui-ci ; comme elles
            peut générer des ramettes de trafic réseau afin que le trafic légitime ne puisse pas passer à travers.
            Les attaques DoS simples sont relativement faciles à mettre en place, il suffit tout simplement 
            de submerger une machine avec des requêtes qui ont l'air d'être légitimes. \cite{virus} %vérifié
        \item[Logiciels malveillants (Malware) :] Ce sont des logiciels dont l'intention est malveillante, 
            ou dont l'effet est malveillant. Le spectre des logiciels malveillants couvre une grande variété 
            de menaces spécifiques, y compris les virus, les vers, les chevaux de Troie et les logiciels espions.
            \cite{virus}%vérifié
    \end{description}

    L'étude dans ce mémoire sera portée sur les Malwares et les techniques employés par ces derniers pour propager
    et cacher leurs présences dans un système. %vérifié

    \subsection{L'illusion de la sécurité absolue}
    Évidemment, tout le monde veut que son ordinateur soit sécurisé contre les menaces. Malheureusement, 
    la sécurité absolue n'existe pas. Vous pouvez prendre beaucoup de précautions techniques pour protéger vos 
    ordinateurs, mais vos protections sont peu susceptibles d'être efficaces contre un attaquant déterminé avec
    assez de ressources. Comme exemple, un attaquant déterminé pourrait conduire un camion à travers le 
    mur de votre bâtiment et voler vos ordinateurs. 
    Pour faire simple, aucun système n'est infaillible, et aucune information n'est à l'abri, c'est juste une
    question de volonté et de ressources. \cite{virus} %vérifié

    Même si aucun système ne peut être sécurisé d'une manière absolue, une sécurité relative peut être considérée
    en fonction de six facteurs : %vérifié
    \begin{itemize}%[label=$\star$]
        \item Quelle est l'importance de l'information qui est protégée ?
        \item Quel est l'impact potentiel, si la sécurité est violée ?
        \item Qui voudra obtenir cette information ?
        \item Quelles sont les compétences et les ressources disponibles à l'attaquant ?
        \item Quelles sont les contraintes imposées par l'usage légitime ?
        \item Quelles sont les ressources disponibles pour mettre en œuvre la sécurité ?
    \end{itemize} %vérifié

    Diviser le concept de la sécurité de cette manière modifie le problème. La sécurité n'est plus une question 
    binaire dont les deux choix possibles sont : sécurisé ou non-sécurisé, mais plutôt une gestion des risques.
    L'implémentation de la sécurité peut être considérée comme le compromis entre le niveau de protection, 
    la facilité d'utilisation du système, et le coût de l'implémentation. \cite{virus} %vérifié

    \subsection{Les objectifs de la sécurité informatique}
    La sécurité informatique consiste en générale à assurer que les ressources matérielles ou logicielles sont 
    uniquement utilisées dans le cadre prévu. La sécurité informatique vise généralement cinq principaux objectifs :
    %vérifié
    \begin{description}
        \item[L'intégrité :] Consiste à déterminer si les données n'ont pas été altérées durant la communication 
            (de manière fortuite ou intentionnelle) \cite{objectif_intrapole} .
        \item[La confidentialité :] Consiste à rendre l'information inaccessible à d'autres personnes que les 
            seuls acteurs de transaction \cite{objectif_comment_ca_marche} .
        \item[La disponibilité :] L'objectif de la disponibilité est de garantir l'accès à un service 
            ou à des ressources \cite{objectif_comment_ca_marche} .
        \item[La non-répudation :] La non-répudation de l'information est la garantie qu'aucun des correspondants 
            ne pourra nier la transaction \cite{objectif_comment_ca_marche} .
        \item[L'authentification :] Consiste à assurer l'identité d'un utilisateur, et bien garantir que chaque
            personne est celui qu'il prétend être \cite{objectif_intrapole} .
    \end{description} %vérifié

    \subsection{Les attaques informatiques}
    Une attaque informatique est une attaque \emph{active} ou \emph{passive} pour l'exploitation des systèmes informatiques, 
    des entreprises, et des réseaux, est-ce par divers moyens d’actes malveillants. 
    Il existe de multiples catégories d’attaques informatiques dont les buts et les techniques différents. %vérifié
    \begin{description}
        \item[Interruption :] Ce type d'attaque vise la \emph{disponibilité}. Il s’agit de compromettre les 
            ressources physiques ou logiques d'un système. Exemple : les attaques \emph{DoS} et \emph{mac flooding}.
            %vérifié
        \item[Fabrication :] Ce type d’attaque vise l'\emph{authenticité}. Il s'agit de l’ajout d’une 
            information contrefait au système. Exemple : l'attaque \emph{ARP cache poisoning}. %vérifié
        \item[Interception :] Ce type d’attaque vise la \emph{confidentialité}. Il s’agit de l'acquisition de données 
            destinées a autrui. Exemple : l’attaque \emph{sniffing}. %vérifié
        \item[Modification :] Ce type d’attaque vise l'\emph{intégrité}. Il s’agit de la modification 
            de données d’un programme pour changer/altérer le fonctionnement ou le comportement de ce dernier. 
            Exemple : attaque par injection de code. %vérifié
    \end{description}
    
\section{Le Hacking}
    Le hacking \cite{bases_hacking} est l'art et la manière de modifier un élément physique ou virtuel 
    de sorte à ce que celui-ci n'ait pas le comportement prévu initialement par ses créateurs. %vérifié

    Démarrer une voiture sans clé est du hacking de la même façon que cracker le mot de passe de la session 
    d'un utilisateur ou accéder à des informations cryptées. Tout ce qui n'était pas prévu par le concepteur et 
    qui a été détourné de toute forme qui soit est du hacking  \cite{hacking} . %vérifié

    Le hacking ne se résume donc pas au cracking wifi du voisin ou à la prise de possession de l'ordinateur 
    d'une victime. %vérifié

    Les hackers sont des personnes bien ou mal intentionnées qui utilisent leurs connaissances pour exploiter 
    des failles dans des environnements distincts afin d'effectuer des modifications entraînant une attitude 
    différente de celle que l'auteur a prévu pour pouvoir en tirer un avantage \cite{hacking} . %vérifié

    Cela peut paraître néfaste ; et ça l'est dans certains cas. Cependant, sans les hackers, internet ne prendrait 
    pas la tournure qu'il est en train de prendre, à savoir une tournure open source et communautaire, 
    déliée des pouvoirs qui tentent de se l'accaparer \cite{hacking} . %vérifié

    \subsection{Différents type de hacker}
    Il existe différents types de hackers dans le monde de la sécurité informatique. On y trouve : %vérifié
    \cite{types_de_hacker}

    \begin{description}
        \item[Les gamins des scripts :] Aussi appelé \emph{Script Kiddie}. 
            Ce sont des personnes qui utilisent des outils, des scripts, et des programmes
            créés par des vrais hackers. Dans des mots plus simples, ce sont ceux qui ne savent pas comment 
            les systèmes marchent, mais qui arrivent quand même à les exploiter en utilisant des outils qui 
            existent déjà. %vérifié

        \item[Les hackers aux Chapeaux Blancs :] Aussi appelé \emph{White Hat Hackers}.
            Ce sont les gars gentils qui utilisent le hacking pour défendre. 
            Le but principal d'un hacker au chapeau blanc est d'améliorer la sécurité des systèmes, en trouvant
            des failles de sécurités et les fixer. Ils travaillent pour des organisations ou individuellement, pour 
            rendre le cyberespace plus sécurisé. 

            Le slogan de cette catégorie de hacker est : \emph{Apprendre l'attaque pour mieux se défendre}. %vérifié
            
        \item[Les hackers aux chapeaux noirs :] Aussi appelé \emph{Black Hat Hackers}.
            Ce sont vraiment les mauvais gars, qui ont des intentions malicieuses. Ils utilisent le hacking 
            pour voler de l'argent, infecté des systèmes avec les Malwares (\autoref{malwares}) \ldots{} etc. %vérifié

        \item[Les hackers aux chapeaux gris :] aussi appelé \emph{Grey Hat Hackers}.
            Ce sont des hackers qui n'ont pas d'intentions malicieuses, mais qui tentent quand même de pénétrer 
            un système dont ils n'ont pas l'autorisation d'accès, juste pour le plaisir ou pour démontrer 
            l'existence d'une faille de sécurité. %vérifié

        \item[Les Hacktivists :] Les hackers appartenant à cette catégorie, utilisent leurs compétences pour 
            dénoncer l'injustice et/ou attaquer le système ou le site web d'une entité responsable d'une injustice. 
            L'un des Hacktivists le plus connu est \emph{Anonymous}. %vérifié
    \end{description}

    \subsection{Les phases d'un hacking réussi}
    Il y a en tout cinq grandes phases, qui lorsqu'elles sont accomplies avec succès, garantissent presque 
    toujours un hacking \cite{bases_hacking} réussi. Le hacker devra donc les couvrir une 
    par une et en tirer des conclusions. %vérifié

    Les cinq phases du hacking sont les suivantes \cite{phases_of_hacking} : %vérifié
    \begin{description}
        \item[La reconnaissance :] C'est la phase primaire où le hacker essaye de collecter autant d'informations que
        possible sur la cible. Ça inclut l'identification de la cible, trouver les adresses IP \cite{reseau}
        de la cible \ldots{} etc. %vérifié

        \item[Le scan :] Ça consiste à prendre les informations découvertes durant la phase de reconnaissance, et 
            les utilisés pour examiner le réseau de la cible. Les outils qui peuvent être utilisés par 
            le hacker durant cette phase sont les scanners de ports et les scanners de vulnérabilités. %vérifié

        \item[Le gain d'accès :] Après le scan, le hacker modélise le plan du réseau de la cible avec l'aide
            des données récoltées pendant les deux phases précédentes. C'est la 
            phase où le vrai hacking se déroule. Les vulnérabilités découvertes durant les phases précédentes sont 
            maintenant exploitées pour gagner un accès à la cible. %vérifié

        \item[Le maintien d'accès :] Une fois le hacker a gagné l'accès à la cible, la prochaine étape consiste à 
            maintenir cet accès pour une future utilisation. %vérifié

        \item[La couverture des traces :] Une fois le hacker a pu gagner et maintenir l'accès, il couvre ses traces 
            pour éviter qu'il soit détecter par les personnels de la sécurité, pour continuer d'utiliser le système.
            Les hackers essayent de supprimer toute trace de l'attaque, comme les fichiers journaux ou 
            les alarmes des systèmes de détections d'intrusions. %vérifié

    \end{description}

    \subsection{Les vulnérabilités}
    Les vulnérabilités sont des points faibles par lesquels l'intégrité d'un système peut être atteinte. Ces points 
    faibles ne sont pas toujours exploitables ; cependant, elles peuvent être combinées pour 
    réussir une exploitation. %vérifié
    
    Les vulnérabilités peuvent être exploitées pour infiltrer un système, ou pour obtenir des privilèges supplémentaires
    sur des systèmes déjà atteints. Elles peuvent être exploitées automatiquement par les Malwares 
    (\autoref{malwares}), ou manuellement par des personnes visant directement un système. %vérifié

    Les vulnérabilités se divisent en deux catégories, selon l'endroit où elles se situent : %vérifié
    
    \begin{description}
        \item[Vulnérabilités techniques :] Elles proviennent souvent de la négligence ou de l'inexpérience d'un 
            programmeur. Une vulnérabilité de cette catégorie permet généralement à un attaquant de duper 
            une application, par exemple en outrepassant les vérifications de contrôle d'accès ou en exécutant
            des commandes sur le système hébergeant l'application.
            \label{buffer_overflow} %vérifié

            Quelques vulnérabilités techniques surviennent lorsque l'entrée d'un utilisateur n'est pas contrôlée,
            permettant l'exécution de commandes ou de requêtes. D'autres proviennent d'erreurs d'un programme
            lors de la vérification des tampons de données (qui peuvent être dépassés), 
            causant ainsi une corruption de la pile mémoire
            (et ainsi permettre l'exécution de code fourni par l'attaquant). \cite{vulnerabilites} %vérifié

        \item[Vulnérabilités humaines :] Les humains sont le maillon le plus faible dans la chaîne de sécurité.
            Les humains peuvent oublier d'appliquer des patchs de sécurité critique, introduire des bugs
            exploitable, configurer mal leurs logiciels au point de les rendre vulnérables \ldots{}
            Il existe même une catégorie d'attaque dédiée à duper les individus, appelée \emph{ingénierie sociale} 
            \cite{bases_hacking}. %vérifié
    \end{description}

\section{Les Malwares} \label{malwares}
    Le mot \emph{Malware} est une abréviation de \emph{\textbf{mal}icious soft\textbf{ware}}, qui signifie 
    \emph{logiciel malveillant}. Un Malware est un logiciel qui peut être utilisé pour compromettre le
    fonctionnement des ordinateurs, voler des informations ou des données, contourner les contrôles d'accès
    \ldots{} etc. Dans la partie qui suit, les types les plus courants des Malwares seront abordés avec une explication 
    de leurs modes de fonctionnement et de leurs objectifs principaux. \cite{malware_types}%vérifié

    \subsection{Les types des Malwares}

    \begin{description}
        \item[Bombe logique :] Une bombe logique est un code constitué principalement de deux parties :
            \begin{enumerate}
                \item Une \emph{charge utile}, qui est l'action à effectuer. La charge utile peut être n'importe quoi,
                    mais puisque c'est un logiciel malveillant, la charge a généralement un effet malicieux.
                \item Une \emph{gâchette}, qui est une condition booléenne qui indique le moment du lancement de la
                    charge. La vraie limite de la charge est limitée seulement par l'imagination, et peut être basé sur
                    des conditions comme la date, le nom de l'utilisateur connecté, la version du système d'exploitation
                    \ldots etc.
            \end{enumerate}
            Une bombe logique peut être insérée dans un code existant, comme elle peut être autonome. 
            Ci-dessous un pseudo code exemple d'une bombe logique insérer dans un code existant :
            \begin{verbatim}
                instructions
                if condition is True:
                    executer_charge()
                instructions
            \end{verbatim}
            Une bombe logique peut être discrète et très difficile à trouver, surtout si elle s'y trouve parmi des 
            millions de lignes de code, et elle peut être utilisée facilement pour causer des 
            dégâts phénoménaux. \cite{virus} %vérifié

        \item[Ver :] Un ver informatique est un programme autonome qui peut s'auto-reproduire, 
            et qui ne repose pas sur d’autres exécutables. Il se propagent d'une machine à travers les 
            réseaux en profitant des ports ouverts, mais la méthode la plus classique consiste à s'introduire 
            sous la forme d'une pièce jointe attachée à un mail.%vérifié

            Un ver ne se multiplie pas localement, contrairement aux virus, mais sa méthode la plus 
            habituelle de propagation consiste à s'envoyer dans des mails générés automatiquement. 
            Ces mails sont expédiés à l'insu de l'utilisateur vers diverses adresses. 
            Ces adresses sont généralement prélevées par le ver dans les fichiers présents sur le disque%vérifié

            Les vers installent généralement sur l'ordinateur d'autres programmes nocifs : comme les logiciels 
            espions, les Keyloggers, et les portes dérobées. \cite{ver_informatique}

        \item[Porte dérobée :] \label{backdoor} Une porte dérobée (aussi appelée Backdoor) est n'importe quel mécanisme qui peut contourner un
            contrôle de sécurité ordinaire. Les programmeurs créent dés fois des portes dérobées pour 
            des raisons légitimes, telles que passer une procédure d'authentification coûteuse lors 
            de débogage d'un service réseau.

            Comme les bombes logiques, les portes dérobées peuvent être insérées dans un code existant, comme elle
            peuvent être autonomes. Ci-dessous se trouve un pseudo code exemple d'une porte dérobée insérer dans 
            un code existant :
            \begin{verbatim}
                nom_utilisateur = lecture()
                mot_de_pass = lecture()
                if nom_utilisateur = "créateur de la porte dérobée":
                    laissee_entree()
                if est_valide(nom_utilisateur) and 
                        est_valide(mot_de_pass):
                    laissee_entree()
                else:
                    renvoyer()
            \end{verbatim}

            Il existe un type spécial des portes dérobées, appelé \emph{RAT}\footnote{Le mot RAT est une
            abréviation de \emph{Remote Access Tool} qui signifie outil d'accès à distance.}. Ce type de portes
            dérobées donne un accès total à un ordinateur pour un contrôle à distance. L'utilisateur peut
            installer un RAT intentionnellement pour accéder à un ordinateur de travail depuis la maison, ou pour 
            permettre à un technicien de l'aider de loin, mais les RAT peuvent aussi être utilisés par les 
            pirates pour des raisons non-légitimes. \cite{virus} %vérifié

        \item[Logiciel espion :] C'est un logiciel qui récolte des informations d’une machine et 
            la transmet a quelqu'un d'autre.\\ %vérifié
            Les récoltes d'informations peuvent ainsi être :
            \begin{itemize}
                \item Les noms d’utilisateurs et mots de passe, Celles-ci pourraient être récoltées à partir de
                    fichiers sur la machine ou en enregistrant ce que les tapes l'utilisateur en utilisant 
                    un \emph{Keylogger}\footnote{Un Keylogger est un type de logiciel espion spécialisé pour 
                    espionner les frappes au clavier sur l'ordinateur qui l'héberge, et pour les transmettre 
                    via internet à une adresse où un pirate pourra les exploiter.}.
                \item Les adresses mail.
                \item Les comptes bancaires et les numéros des cartes de crédit.
                \item Clés de licence des logiciels.
            \end{itemize} %vérifié
            Un logiciel espion peut arriver sur une machine de diverses manières, par exemple avec un logiciel 
            que l’utilisateur installe ou l’exploitation des failles techniques dans les navigateurs web ce 
            qui causera l’installation des logiciels espions simplement en visitant une page web.\cite{virus}%vérifié

        \item[Logiciel publicitaire :] Ce logiciel est un type de Malware qui livre automatiquement des 
            annonces ou des publicités. C’est le moins dangereux et lucratif des logiciels malveillants. %vérifié

        \item[Logiciel de rançon :] C’est un logiciel qui prend en otage un système en demandant une rançon 
            du propriétaire. Ce type de logiciels malveillants, généralement, crypte les données du 
            disque dur et affiche un message d’alerte informant l’utilisateur qu’il faut payer une 
            rançon pour que les données puissent être récupérées. Le message contient aussi les étapes à suivre 
            pour payer la rançon, comme il contient un avertissement expliquant que toutes les données 
            seront effacées et perdues à jamais si l’utilisateur tante d'appeler la police, ou ne paie pas la rançon 
            dans le délai précisé par le message. %vérifié

        \item[Cheval de Troie :] Dans le monde de l'informatique, un cheval de Troie est un type de logiciel
            malveillant, souvent confondu avec les virus ou autres Malwares. Le cheval de Troie est un logiciel
            en apparence légitime, mais qui contient un logiciel malveillant. Le rôle du cheval de Troie est de faire
            entrer ce parasite sur l'ordinateur et de l'y installer à l'insu de l'utilisateur.

            Le programme contenu est appelé la \emph{charge utile}. Il peut s'agir de n'importe quel type
            de logiciel malveillant (virus, logiciel espion \ldots). C'est ce logiciel malveillant, et lui
            seul, qui va exécuter des actions au sein de l'ordinateur victime. Le cheval de Troie n'est rien
            d'autre que le véhicule. Il n'est pas nuisible en lui-même, car il n'exécute aucune action,
            si ce n'est celle de permettre l'installation du vrai logiciel malveillant. 
            \cite{wikipedia_trojan} %vérifié

        \item[Bot :] Les bots sont des logiciels créés pour effectuer des tâches automatiquement. 
            Tandis qu’il y a quelques bots qui sont créés à des fins relativement inoffensives 
            (comme pour les jeux vidéos), il devient de plus en plus fréquent de voir des bots utilisés malicieusement. 
            Les bots peuvent être utilisés dans plusieurs activités plus ou moins malicieuses, on y trouve :
            \begin{itemize}
                \item Le contrôle d'un ensemble d’ordinateurs pour réussir les attaques d’interruption (DDoS
                    \footnote{(Distributed Denial of Service) C'est une attaque DoS qui implique beaucoup de
                    participants, ce qui rend l'attaque très difficile à arrêter.}) facilement ;
                \item La collecte des adresses mail sur les pages web ;
                \item Le balayage des données sur les serveurs ;
            \end{itemize}
            Les sites web peuvent se protéger contre les bots en utilisant le système de 
            CAPTCHA\footnote{Un système qui demande de l'utilisateur de faire entrer des informations visuelles ou
            auditives avant l'envoi d'une requête, pour assurer que c'est un utilisateur humain.} qui vérifie que 
            les utilisateurs qui utilisent les services du site sont belle et bien des humains. 
            \cite{malware_types} %vérifié

            \item[Virus :] Comme la majeure partie de cette étude concerne ce type de Malware, 
            le prochaine chapitre lui sera consacré.
    \end{description}

\section{Conclusion}
À la fin de ce premier chapitre, certaines notions sur le domaine de la sécurité informatique de manière générale, et sur le Hacking et les programmes malveillants sont désormais plus claires. 
Leur compréhension est indispensable pour la suite de l'étude. %vérifié
